\documentclass[11pt]{article}

% --- Packages ---
\usepackage[utf8]{inputenc}
\usepackage[T1]{fontenc}
\usepackage{lmodern}
\usepackage{amsmath, amssymb, amsthm}
\usepackage{geometry}
\usepackage{hyperref}
\usepackage{bm}

\geometry{margin=1in}

% --- Theorem environments ---
\newtheorem{theorem}{Theorem}
\newtheorem{lemma}{Lemma}
\newtheorem{corollary}{Corollary}

% --- Title ---
\title{\textbf{SERC v2.1 -- Mathematical Foundations}}
\author{Leszek Papiernik}
\date{February 2026}

\begin{document}

\maketitle

\begin{center}
\large Stability of Relational Dynamics on the Simplex
\end{center}

\vspace{1em}

\begin{abstract}
This document provides the mathematical foundations of the SERC framework.
We analyze the geometry of the 3-simplex, the Gram metric $G = 4I - J$,
the projected gradient flow, and prove global asymptotic stability of the
relational equilibrium $P_0 = \tfrac14 \mathbf{1}$.
We also derive the exponential relaxation rate and interpret the results
in the context of LLM guardrails.
\end{abstract}

\section{Stability of Relational Dynamics on the Simplex}
\label{sec:stability}

\subsection{Setup}

Let $\Delta^3 \subset \mathbb{R}^4$ denote the standard 3-simplex:
\begin{equation}
  \Delta^3 = \bigl\{ Z \in \mathbb{R}^4_{\geq 0} \;\big|\; \mathbf{1}^\top Z = 1 \bigr\}.
\end{equation}
The tangent space at any interior point is
\begin{equation}
  T\Delta^3 = \bigl\{ v \in \mathbb{R}^4 \;\big|\; \mathbf{1}^\top v = 0 \bigr\},
\end{equation}
with orthogonal projector onto $T\Delta^3$:
\begin{equation}
  P = I_4 - \tfrac{1}{4}\,\mathbf{1}\mathbf{1}^\top.
\end{equation}

The \emph{relational tension functional} is
\begin{equation}
  \Omega(Z) = \tfrac{1}{2}\,Z^\top G Z,
  \qquad G = 4I_4 - J_4,
\end{equation}
where $J_4 = \mathbf{1}\mathbf{1}^\top$. Its gradient is $\nabla\Omega(Z) = GZ$.

\subsection{Spectral Analysis}

\begin{lemma}[Spectrum of $G$ on $T\Delta^3$]
\label{lem:spectrum}
The matrix $G = 4I_4 - J_4$ has eigenvalues $\{0, 4, 4, 4\}$.
The zero eigenvalue corresponds to eigenvector $v_0 = \tfrac{1}{2}\mathbf{1}$,
which is orthogonal to $T\Delta^3$.
The restriction $G|_{T\Delta^3}$ is positive definite with
$\lambda_{\min}(G|_{T\Delta^3}) = 4$.
\end{lemma}

\begin{proof}
Direct computation: $G\mathbf{1} = (4I_4 - J_4)\mathbf{1} = 4\mathbf{1} - 4\mathbf{1} = 0$.
For any $v \perp \mathbf{1}$ (i.e.\ $v \in T\Delta^3$):
$Gv = 4v - J_4 v = 4v - (\mathbf{1}^\top v)\mathbf{1} = 4v$.
Hence $G|_{T\Delta^3} = 4I|_{T\Delta^3}$.
\end{proof}

\subsection{Projected Gradient Flow}

Define the \emph{projected gradient flow} on $\Delta^3$:
\begin{equation}
  \dot{Z} = -P\,\nabla\Omega(Z) = -PGZ.
  \label{eq:flow}
\end{equation}
Since $P$ projects onto $T\Delta^3$ and $\mathbf{1}^\top Z(0) = 1$, we have
$\mathbf{1}^\top\dot{Z} = 0$ for all $t$, so $Z(t) \in \Delta^3$ for all $t \geq 0$.

\begin{theorem}[Lyapunov Stability of $P_0$]
\label{thm:lyapunov}
The point $P_0 = \tfrac{1}{4}\mathbf{1}$ is the unique equilibrium of \eqref{eq:flow}
in $\Delta^3$, and is globally asymptotically stable.
$\Omega$ is a Lyapunov function for this flow.
\end{theorem}

\begin{proof}
\textbf{Step 1: $\Omega$ is non-increasing.}
\begin{equation}
  \frac{d}{dt}\Omega(Z(t))
  = \nabla\Omega(Z)^\top \dot{Z}
  = (GZ)^\top(-PGZ)
  = -(GZ)^\top P(GZ).
\end{equation}
Let $w = GZ$. Since $P$ is an orthogonal projector, $w^\top Pw = \|Pw\|^2 \ge 0$.
Thus $\dot{\Omega} \le 0$, with equality iff $Pw = 0$.

\textbf{Step 2: Unique equilibrium.}
$\dot Z = 0$ iff $PGZ = 0$, i.e.\ $GZ = c\mathbf{1}$.
Write $Z = P_0 + v$ with $v \in T\Delta^3$.
Then $GZ = Gv$ and $Gv = c\mathbf{1}$.
But $Gv \in T\Delta^3$ and $\mathbf{1} \perp T\Delta^3$,
so $c = 0$ and $Gv = 0$.
Since $G|_{T\Delta^3} = 4I$, we get $v = 0$.

\textbf{Step 3: Asymptotic stability.}
$\Omega(Z) \ge 0$ with equality only at $P_0$.
LaSalle's invariance principle gives global asymptotic stability.
\end{proof}

\begin{lemma}[Quadratic lower bound]
\label{lem:quadratic}
For $Z \in \Delta^3$ and $v = Z - P_0 \in T\Delta^3$:
\begin{equation}
  \|PGZ\|^2 = 16\|v\|^2 = 32\,\Omega(Z).
\end{equation}
\end{lemma}

\begin{proof}
Since $Gv = 4v$ on $T\Delta^3$, we have $\|Gv\|^2 = 16\|v\|^2$.
Also $\Omega(Z) = \tfrac12 v^\top Gv = 2\|v\|^2$.
\end{proof}

\begin{corollary}[Exponential Relaxation Rate]
Along trajectories of \eqref{eq:flow}:
\begin{equation}
  \Omega(Z(t)) \le \Omega(Z(0))\,e^{-32t}.
\end{equation}
\end{corollary}

\begin{proof}
From the previous lemma:
\begin{equation}
  \dot\Omega = -\|PGZ\|^2 = -32\,\Omega(Z).
\end{equation}
Integrating gives the result.
\end{proof}

\subsection{Interpretation for LLM Guardrails}

Under the ideal flow, $\Omega(Z(t))$ decreases monotonically to zero.
Actual LLM trajectories follow a discrete perturbed update:
\begin{equation}
  Z_{t+1} = Z_t + \Delta_t.
\end{equation}

Define the relational anomaly:
\begin{equation}
  a_t = \Omega(Z_{t+1}) - \Omega(Z_t).
\end{equation}

Healthy dynamics: $a_t < 0$.  
Pathology: sustained $a_t > 0$.

Guardrail condition $\Omega_{\text{mean}}(W) > \theta$
detects windows where $\sum a_t > 0$.

\end{document}
